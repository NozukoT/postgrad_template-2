
\chapter{Literature Review}
\label{chap:lit_review}

This research work brings an intersection of big data analytics outcomes stemming from an e-government communication strategy using internet technologies in a form of social media platforms amongst others for communicating with citizens.  \\
Therefore, literature survey will focus on the two areas, starting to examine e-government communication phenomenon that has evolved overtime in the public sector, followed by social media platforms and big data as an indirect consequence, and concluding by developments in text mining opportunities (machine learning) in the tax environment from big data generated from e-government communication (internet) platforms.  

\subsubsection{e-government}
[E-GOVERNMENT FUNDAMENTALS] E-government can be known by different terms Electronic Governance, Digital Government, Online Government, e-Gov, Electronic Government, "is means to offer services to those within its authority to transact electronically within the government."\\
With majority of citizens using smartphones and access to the internet for applications the digital government innovation is inevitable, [book: ch 22 Thomas Jowazski - From electronic government to policy driven electronic governance evolution of technology use in government], even though "Electronic government is not focused on technology but utilizing technology to improve organization environment within government, on transforming the internal working of government through technology"].\\

[Source: EY Connected Citizen Survey 2020, EYGM Limited 2020.]  In a survey, percentage of citizens who think technology will change, for the better, the way they conduct different tasks - "57percent reported of the entire sample think technology will change, for the better, the way they conduct different tasks.].  In the same study, for the South African sample, 57percent has reported to think that  government and public services are effective in using digital technology to respond to the COVID-19 pandemic?\\
[Making e-gvt work: Adopting the network approach Government Information Quarterly 31 (2014) 327-336], Suggests e-gvt success to use a network explained in 4 critical factors, 1) "Ensuring Availability of resources for success of e-government project", "Ensuring consensus about roles, goals and responsibilities of each other, "Ensuring that electronic process are not replaced/tampered in the long run", "Ensuring smooth functioning of inter-linked processes and facilitating coordination by seamless flow of information", "Ensuring long term commitment of all actors necessary for smooth functioning of the electronic process."\\

[To read Data.gov] and [Government data invisible hand]
The existence of effective interactive participation platforms through social media largely depends on the role played by the public administrator, who may be neutral or a dynamic advocate of citizen participation (Bonsón et al., 2017).\\ 
Public administrators need to mobilise citizens to participate in public issues, collabo-rating in the decision-making process and proposing solutions to the problems of govern- ment (Bonsón et al., 2017; Mergel, 2013b; Zheng and Zheng, 2014), but to achieve this outcome, public administrators must interact with citizens to identify the problems to be addressed so that appropriate solutions may be found and applied (Bonsón et al., 2017; Zafiropoulos et al., 2014).\\
Similarly, Mergel (2013a) observed that attaining a high level of engagement between the society and government means the latter must go beyond the mere publication of content; citizens must be encouraged to comment on government posts to social net- works and to take an active part in this field. In this respect, however, Bonsón et al. (2017) found no relationship between the level of government activity in social media and citizen engagement and suggested that an increase in the number of government posts in channels such as Facebook and Twitter does not necessarily produce higher levels of citizen engagement.\\

Having said that it is also imperative for governments to find innovation to observe and monitor impact of interactive engagements attention with the citizens or users. \\ 

[5]As early as year 2017 the South African government embarked on social media use, after careful consideration of associated risks and governance aspects for legislation providing guidelines to comfortable implement social media technologies, with implementation amongst its departments without a well planned e-government communication strategies. However, [6], this inability of government organizations to go beyond the information broadcasting phase has been highlighted in research (Haro-de-Rosario et al., 2018; Zavattaro, Sementelli, 2014).\\

[Assessing South African Government?s Use of Social Media] This PDF article assesses the use of social media by the South African government for citizen participation. The researchers conducted a qualitative study analyzing the social media pages of South African provinces and municipalities. They found that while all provinces and municipalities have social media accounts, they are mainly used for information broadcasting rather than meaningful engagement and participation. The study highlights the need for a strategy that enhances citizen participation through social media. The researchers categorized the social media use into democratic processes, participation areas, participatory techniques, categories of tools, and technologies. They identified areas such as information provision, service delivery, discourse, consultation, crisis/emergency management, and community building. However, they note that the current use of social media by the South African government is mainly superficial and lacks true public participation. The article concludes by recommending that government organizations post relevant content, actively engage with citizens, and explore the full potential of social media for citizen participation in South Africa.\\
[1] As of January 2021, globally, the active social media users reached 4,29 billion points, and due to increase ownerships of smart devices for the African population of 1,32 billion has 217,5 millions active social media users., despite digital divide for Countries with Developing Economies (CDEs). The digital 2021 report reveals that “ a portfolio approach to social media may help improve efficiency and effectiveness”, an important point for a public administration use of social media.  [3] The e-government initiative is well established and has steadily progressed in a robust two-way communication with citizens.  While effective use of social media efforts in the public sector lacks significant meaningful approach as compared to the private sector, monitoring and evaluation of these efforts is also an important factor for effective and effective strategy to manage efforts also for the public sector. \\
[2]For public administration e-participation amongst others includes social media promoting a two-way communication to transactional between governments and citizens in real time.\\  However, social media effectiveness in government engagement is an area that needs an exploration and investigation for factors such as bureaucracy and overall government ICT policy and strategy can hinder social media purposes and user - government and overall underlying values governing government department ethos.\\
[Chapter 4 Measuring the impact of social media use in the public sector]\\

\begin{table}
    \centering
    \begin{tabular}{cc}
        E-Government & Social Media Use \\
         Static & Bi-directional, interactive\\
         Push information & Pull information\\
         Single author (usually Websmaster) & Multiple factor (providing agency and members of their audiences) \\
         Information or - transaction - focused & Interaction-focused\\
         Large financial investments & Applications created by third parties - free use\\
    \end{tabular}
    \caption{Difference between e-government and social media use in the public sector}
    \label{tab:my_label}
\end{table}

\subsubsection{Social media technologies as e-participation tools in government}

Social media technologies have become essential components to create a two way communication between citizens and government at no costs, as a build up to big data.  

[Gvt as a platfform - Tim O'Reilly], Government 2.0 is the use of technology especially collaborative technologies to better address collective problems facing countries in all levels or spheres of government.  --  This PDF document is an essay written by Tim O'Reilly about the concept of "Government 2.0" and how technology can be used to improve government systems. O'Reilly discusses the power of harnessing the creativity and collaboration of individuals, as seen in successful companies like Google, Amazon, and Wikipedia. He argues that government should adopt a similar approach by using technology to better address collective problems and engage citizens. O'Reilly emphasizes the importance of open standards, simplicity, and designing for participation in creating a successful government platform. He also highlights the value of learning from users and adapting to their needs, as demonstrated by the success of applications like Google Maps. Overall, O'Reilly encourages government to embrace a more innovative and participatory model to improve its services and foster citizen engagement.

[Do E-government Services ‘Really’ Make Life Easier? Analyzing Demographic Indicators of Turkish Citizens’ E-government Perception Using Ordered Response Models, Doi:10.5901/mjss.2015.v6n1p185], while collaborative technologies necessary the question is do they make life easier, so the study demonstrates that effective e-government should promote interaction openly between government and customer.  For continuity, [Digital Government Evolution: from Transformation to Contextualization], Explains the process of expanding digital government in four phases "Digitization", "Transformation", "Engagement" and "Contextualization" thus as an idea for a possible approach. 

"One of key skills required of both technologies and gvt officials is how best to aggregate public opinion or data produced by public actions to reveal new information or patterns" Andy Oram  -- As an opening line even for the abstract.  The importance to have a hashtag recommendation for aggregating granular data into context at the inception level.  

[Big Data and Artificial Intelligence in Policy Making: A Mini- Review Approach] articulates entry and benefits of big data and Artificial Intelligence analytics providing new patterns that opens the closed public sector enhancing communication with customize and on time, rich insights for policy making process.  

[Engaging the Public in Ethical Reasoning About Big Data], Effects of ethical consideration and privacy concerns affect both private (profit/financial interest) and public sector to uphold trust during use of these technologies for order and transparency covering governance.  While  [The complexity of public and private policies for big data], suggests four key elements of the big data policy formulation within the European context, namely, ""Reverberation, nature of expertise, use in backroom deals and changing character and indeed confounding of public and private responsibilities".  

[What does Big Data mean to public affairs research? Understanding the methodological and analytical challenges Big Data for the public sector]The PDF discusses the concept of Big Data in the context of public affairs research. It clarifies the definition of Big Data and highlights its potential as well as limitations in policy making.  The authors emphasize the need for public sector practitioners and researchers to consider methodological, ethical, and analytical challenges when utilizing Big Data. They note that while Big Data offers real-time insights, it may also raise privacy concerns and result in biased predictions due to certain demographics being underrepresented. The authors propose that combining Big Data with administratively collected data and smaller datasets can enhance public programs. However, they stress the importance of understanding the limitations and potential misuse of Big Data.

[Engaging the Public in Ethical Reasoning About Big Data]  -- The PDF document discusses the importance of engaging the public in ethical reasoning about big data. The author emphasizes that the public plays a critical role in privacy and ethical issues related to big data research, and scientists must take public concerns seriously and build trust in specific projects. The chapter explores examples of engaging the public in Wikipedia and contrasts it with "Notice and Consent" forms. The author suggests that scientists should adopt best practices in protecting big data, be transparent about data management practices, make smart choices in deploying digital solutions, and demonstrate a strong commitment to privacy and data security. The document also highlights the ethical concerns in big data and the need for professionals to adapt to new ethical issues related to digital privacy rights. It discusses the implications of the Code (or Architecture), Laws, Markets, and Norms in shaping online activities, and the importance of trust in the efficacy of institutions. The author further explores the public perception of big data, privacy, and digital civil liberties, and the challenges in understanding complex tools for protecting privacy. It emphasizes the need for push technologies and examples of online service providers that have taken steps to improve privacy. The chapter also discusses legal challenges, market-based solutions, and the role of norms in engaging the public. Finally, it presents some practical suggestions for researchers, such as adopting best practices in code and data management, using alternative technologies, and emphasizing the human element in big data research.

[Erschienen in: Public Administration Review ; 76 (2016), 6. - S. 928-937 https://dx.doi.org/10.1111/puar.12625]
Articulates opportunities for real time availability of big data directly from citizens, objectively to enhance immediate operations response by public sector and acting upon citizens needs, relying on capability of data analytics results to open new areas of data insights.  [The Big Data Analysis and Visualization of Mass Messages under “Smart Government Affairs” Based on Text Mining], in this recent study demonstrating public sector use of big data analytics for unstructured of social media sources providing new insights to understand and respond timeously to citizens problems, thus efficiency in government operations to citizens promoted.

[https://doi.org/10.1155/2021/9936217, Research on the Design of Government Affairs Platform in the Context of Big Data.]

The PDF document discusses the design of a government affairs platform in the context of big data. It explores the concepts and theories of big data and analyzes the challenges faced by Chinese government management under the impact of big data. The paper suggests that governments should seize the opportunities that big data brings in terms of management efficiency, administrative capacity, and public services. The design of the government affairs platform is based on service-oriented architecture (SOA) and web services technology. The deep learning algorithm is used to construct a monitoring platform for real-time monitoring of government behavior and analyze the government's intention behavior. The PDF also describes the implementation of the platform using SOAP communication protocol, XML data description, and security measures. Overall, the PDF provides insights into the application of big data in government management and the design of a government affairs platform.

[14] While, social media platforms are instrumental to the production of user generated unstructured data aiding to the big data ecosystem, there are huge  benefits for organisations that have abilities in processing of real-time, short length with appropriate text analytics approach.

[2, 9] Suggest that in order to have a fully utilized government as a strategy for using social media, it is important to anchor this approach with frequent monitoring of effectiveness use of social media.  The social media platform offers dynamic performance that is constantly improving, such as quantity of users, time factors for real-time communication, metadata available due to IoT and increasing content of social media data in real-time.  [14] Introduces a method for effectiveness measurement, a content based predictive analytics recommender which works well with tagged featured data, such as systems with “like” or “dislikes” options.  [15] However, in the absence of tagged features an alternative is to “model a framework designing characteristics to generate user profiles, with aided advantages of semantic entity- and topic-based user modeling strategies”.  

[3] From a view of social networks being instrumental to fullfil a strategic communication purpose for organizations,  through constant monitor of emotional barometer with current, while predictions streamline investments channeling emotions for new users in real-time.  However, lack of research and capabilities of big data infrastructure to store, process, analytics for handling both real-time and batch input datasets for real-time output.

A point of social media analytics gaining  momentum, more effort is also required in developing social media intelligence as a catalyst to provide organizations with context rich frameworks and mechanisms for actionable decision-making [9].   [9, 10] Towards achieving a context rich framework is a social media competitive intelligence, which is a framework that can be used for instance to identify industry trends in comparison to a competitor’s social media performance to benefit specific business operations, in this case this would be for an overall monitoring and evaluation of communication performance in relation to others.  [7] It follows that when social networks are used ethically can be useful to cover various aspects such as, obtain competitive advantage, listening  to users, while participating in conversation with users, shaping relationships from a meaningful content and insights that offers an impactful value appropriation.  

[11] Suggests therefore, Twitter data can enhance required real-time social media response similarly to the application of domain adaption classifiers such as those cover emergency events for disaster response beforehand.   The domain adapted classifiers use machine learning supervised techniques that learns classifiers from both unlabeled targeted data and source label data an  approach uses a Naïve Bayes classifier. 

Overall, text mining techniques can be done through classification, topic modeling, not forsaking sentiment analysis, regression techniques to help understand customers views [17]. 

View from tax authorities

[Comprehensive review of text-mining applications in finance]  -- This PDF article provides a comprehensive review of text-mining applications in finance. The authors discuss the recent literature on text-mining applications in financial forecasting, banking, and corporate finance. They analyze the existing literature on text mining in financial applications, summarize recent studies, and briefly discuss various text-mining methods being applied in the financial domain. The article also highlights the challenges faced in these applications and discusses the future scope of text mining in finance. Key areas covered include sentiment analysis, information extraction, natural language processing, text classification, and deep learning. The authors provide examples of how text mining has been used in financial predictions, such as stock market and forex trends, as well as in banking applications like money laundering prevention. The article concludes with a tabular summary of additional research studies in the field.

Any measure for a return on investment amongst others for social media use are benefits from big data analytics outcomes in government departments including the tax authorities.   Meaning the challenge for discovery of metrics for new data patterns is relevant, and continuously for evaluation of impact for service delivery efforts directly affecting them and cannot be solved elsewhere as public monopolies.   
[4] Since revenue authorities deployment of social media technologies in 2010 with few that had started, however, there is a business sense evident from previous studies that use of social media since it started in 2010 [survey covering 26 revenue authorities to reveal that  measuring effective use of social media] includes metrics such “as the approaches used by revenue bodies to gauge effectiveness were largely confined to numbers of users, visits, views, etc”.  However, some of the suggestions factors to measure:
“Identify influential Twitter followers and number of followers“ for possible presence of visibility to promote two-way communication on relevant “tax-matters”.

“Language and Tweet structure“: Attention to Language and Tweet Structure: Jay Baer points to how social media analytics tools can be useful in evaluating which type of Tweets are successful, the metrics being re-tweets. He suggests that the tools Twitalyzer35 or bit.ly36 data be deployed to see what sort of patterns emerge: Are longer tweets, shorter tweets or those with links most successful? According to Baer on his blog: ―It has (been) found that tweets with links are RT‘d substantially more than tweets without links. 
Tweets asking for help? Know what‘s worked for you in the past, and try to model your future tweeting to mimic it (within reason). 

Repeat your Tweets: Jay Baer states on his blog 37………… ―Yes, it‘s unpopular with the social media purists, but (since) 94percent of tweets are RT‘d within the first hour you need to tap into multiple Twitter audiences throughout the day. I tweet my blog posts 3 times daily, with a different headline each time. This becomes even more important if you have followers in many time zones‖ …… as some revenue bodies do.
\section{The First Section}
\label{sec:second:first_sec}

%%%%%%%%%%%%%%%%%%%%%%%%%%%%%%%%%%%%%%%%%%%%%%%%%
%%%%%%%%%%%%%%%%%%%%%%%%%%%%%%%%%%%%%%%%%%%%%%%%%

\section{The Second Section}
\label{sec:second:second_sec}

%%%%%%%%%%%%%%%%%%%%%%%%%%%%%%%%%%%%%%%%%%%%%%%%%

\subsection{A Subsection}
\label{sec:second:second_sec:one}

%%%%%%%%%%%%%%%%%%%%%%%%%%%%%%%%%%%%%%%%%%%%%%%%%

\subsection{Another Subsection}
\label{sec:second:second_sec:two}

%%%%%%%%%%%%%%%%%%%%%%%%%%%%%%%%%%%%%%%%%%%%%%%%%
%%%%%%%%%%%%%%%%%%%%%%%%%%%%%%%%%%%%%%%%%%%%%%%%%

\section{Summary}
\label{sec:second:summary}

%%%%%%%%%%%%%%%%%%%%%%%%%%%%%%%%%%%%%%%%%%%%%%%%%
%%%%%%%%%%%%%%%%%%%%%%%%%%%%%%%%%%%%%%%%%%%%%%%%%