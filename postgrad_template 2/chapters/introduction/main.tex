
\chapter{Introduction}
\label{chap:introduction}
\pagestyle{headings}
\pagenumbering{arabic}
\setcounter{page}{1}

\begin{quote}
	{\it You may place a brief sketch of the scenario that inspired the research here (you may also choose to leave it out). For example, if you are writing about ACOs, you might write something about the simplicity and efficiency of an ant colony. This may be relatively informal (but not colloquial). Avoid references here. Refer to some of the dissertations on the CIRG website for some ideas on what you might include.}
\end{quote}
Place some very general background information here, setting the scene for where your work fits in tot he broader scheme of things.

For instance, give a very broad overview of the field of CI-based function optimisation. You should already provide some references (here's an example of a reference \cite{ref:Engelbrecht:2002}).

Government utilization of social media has emerged as a potent avenue for fostering a digital footprint, enhancing communication tools, and optimizing service delivery through impactful engagement. Concurrently, this engagement indirectly yields substantial big data, which, when subjected to machine learning prediction techniques, holds the potential to significantly inform operational strategies and policy-making processes. The subsequent exploration, titled 'Machine Learning Algorithms for Social Media Analytics - A Survey,' seeks to unveil latent data patterns through social media text analytics. These patterns, once discerned, stand poised to play a pivotal role in decision-making within the public sector, offering effective support to government managers, influencing policy changes, and contributing to the formulation of innovative strategies.

One of the key focal points of our investigation revolves around the extraction of valuable insights from social media messages, with Twitter data serving as a prime reservoir. Analyzing these messages through text mining techniques provides a rich source of information, shedding light on emergent data patterns that can significantly impact decision-making processes within the public sector. The study also underscores the proactive measures taken by government entities, establishing robust analytics capabilities to unearth business opportunities while concurrently fortifying their social media strategy to comprehend consumer behavior.

The research methodology incorporates a novel approach, focusing on a government department that adopted Twitter as part of its social media technologies in 2018. The objective is to implement an organized strategy, managing and responding to posts guided by recommended topics and hashtags. This approach facilitates effective engagement with citizens, thereby enhancing communication strategy on Twitter. A noteworthy contribution of our study lies in the development of a recommendation algorithm for topic tracking, specifically designed to enhance the newly established Twitter strategy.

Furthermore, the research underscores the potential of social media analytics, specifically for government engagement and effective utilization of social media technologies, exemplified through a case study involving a tax authority. The study is structured to unfold with a literature review, tracing the evolution of government communication strategies from static information platforms to interactive technologies. This sets the stage for an in-depth exploration of data sources, followed by exploratory data analysis and topic modeling techniques to unravel trends and sentiments from tweets and associated hashtags.

The culmination of our research methodology lies in the development of a text classification system for predicting hashtag recommendations. This aids in topic detection and categorization, providing a comprehensive framework for enhancing social media engagement and gaining profound insights into user preferences within the dynamic digital landscape. As we delve into the subsequent chapters, our endeavor is to critically evaluate the results, draw meaningful conclusions, and propose avenues for future enhancements, including the incorporation of real-time data and supplementary information for a more holistic understanding."

\section{Motivation}

As governments increasingly leverage social media for citizen interaction, a wealth of big data becomes available for mining, presenting an opportunity to unveil patterns with real-world applications. This study is motivated by the need to support a focused e-government communication strategy within a tax environment. The goal is to develop long-term effective communication strategies that guide practitioners, embedding responsive user experiences through insights-driven decision-making to enhance brand and reputation. The study aims to reinforce the role of social media as an e-government tool, utilizing big data outcomes to provide analytical opportunities and insights for optimal communication strategies/frameworks between the government and citizens.

The exploratory data analysis (EDA) process reveals a limitation in the little usage of tweets with hashtags, hindering metadata availability for contextualizing tweets and identifying relevant topics. This limitation prompts the proposal of an insight-driven public sector framework to promote effective social media user engagement. The framework aims to predict and categorize tweet contexts for easy identification and early detection of sentiments by predicting hashtags. Recognizing the crucial role of hashtags in addressing this challenge, the study proposes a framework for effective categorization of topics (messages) through the Twitter platform.

\subsection{Problem Statement}
How can big data analytics techniques accurately predict the contents of tweets for the categorization of messages, enabling timely processing for efficient government operations, informed decision-making, and effective responses?

Therefore, the study attempts to answer the following research questions:

\begin{itemize}
    \item Sub-Problem 1:  How effective is exploratory data analysis of user data in unveiling new data patterns that contribute to measures of Twitter engagement, specifically in engaging citizens for policy-making, service delivery, or decision-making?
    \end{itemize}

\begin{itemize}
    \item Sub-Problem 2:  What qualitative insights can be derived from the results of topic modeling regarding trends of topics discussed in government-citizen communication, and how do these insights contribute to understanding sentiment outcomes?
    \end{itemize}

\begin{itemize}
    \item Sub-Problem 3:Can machine-learning techniques accurately predict hashtag recommendations for tax-related metadata, enhancing the categorization of tweet contexts for more effective communication strategy management?
\end{itemize}

In addressing these research questions and sub-problems, the study aims to provide a comprehensive understanding of how big data analytics, exploratory data analysis, and machine-learning techniques can contribute to the enhancement of government-citizen communication on social media platforms, specifically Twitter.

\section{Summary}

The research methodology incorporates a novel approach, focusing on a government department that adopted Twitter as part of its social media technologies in 2018. The objective is to implement an organized strategy, managing and responding to posts guided by recommended topics and hashtags. This approach facilitates effective engagement with citizens, thereby enhancing communication strategy on Twitter. A noteworthy contribution of our study lies in the development of a recommendation algorithm for topic tracking, specifically designed to enhance the newly established Twitter strategy.\\


